\chapter{Introduction}
\label{cp:introduction}
Welcome to the introduction of your dissertation. The introduction of a dissertation serves as a critical component, setting the tone and laying the foundation for the entire research endeavour. It is tasked with providing a clear and concise overview of the research topic, elucidating the context and significance of the study within the broader academic landscape. A well-crafted dissertation introduction should delineate the research problem or question, offering a rationale for its relevance and addressing any existing gaps in knowledge. Furthermore, it typically outlines the objectives and aims of the study, guiding the reader through the anticipated contributions and outcomes. In addition, the introduction often encapsulates the methodology employed, presenting the chosen approach and rationale behind it. Lastly, it functions as a road-map, offering a brief glimpse into the structure and organisation of the dissertation, thereby orienting the reader and facilitating comprehension of the subsequent chapters. Overall, a dissertation introduction should engage the reader's interest, provide a clear framework for the research, and justify its importance in the academic realm. For a clearer and reader-friendly experience on referencing chapters, please refer to the chapter titled \nameref{cp:citations} (referred to as \autoref{cp:citations}).
